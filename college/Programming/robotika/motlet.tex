\documentclass{article}
\usepackage[tmargin=2cm, bmargin=2cm, lmargin=3cm, rmargin=3cm, a4paper]{geometry}

\begin{document}

% awikwok
\par{Yth. Manajer Perekrutan,}
\par\null\par
\par{Saya menulis surat motivasi ini untuk mengekspresikan minat saya dalam divisi Programming di BANYUBRAMANTA.}
\par\null\par
\par{Saya Nama, berusia 17 tahun. Sebagai mahasiswa baru semester I Teknik Elektro di Institut Teknologi Sepuluh Nopember, saya yakin bahwa keterampilan dan pengetahuan yang saya peroleh selama 2 bulan praktikum Dasar Pemrograman akan memungkinkan saya untuk membuat program dengan efisien.}
\par\null\par
\par{Secara khusus, saya memiliki pengalaman membuat program dengan Bahasa Pemrograman C pada Praktikum Dasar Pemrograman Teknik Elektro. Saya juga berhasil memenuhi sebagian besar syarat dari program Maze Robot pada KPP Divisi Programming UKM Robotika.}
\par\null\par
\par{Minat utama saya dalam mengambil divisi Programming di BANYUBRAMANTA berasal dari Video Profile, pada bagian Programming. Saya ingin belajar lebih lanjut tentang framework, bahasa pemrograman (Python) yang digunakan, dan integrasi dengan controller pada robot di BANYUBRAMANTA, serta teknologi-teknologi lain yang tidak biasa digunakan seperti kabel LAN RJ-45 dan kamera waterproof.}
\par\null\par
\par{Minat utama saya juga ada pada Guidebook Open Recruitment, tepat dalam membuat dokumentasi yang komprehensif untuk program robot yang dibuat. Saya bisa membantu mendokumentasikan program yang dibuat.}
\par\null\par
\par{Terima kasih atas peninjauan aplikasi saya dan pertimbangan saya sebagai kandidat untuk divisi Programming di BANYUBRAMANTA. Saya berharap untuk segera mendengar kabar dari Anda.}
\par\null\par
\par{Dengan rasa terima kasih,}
\par{Nama}
\par{Nomor Telepon}
\par{Email}

\end{document}